\documentclass[12pt]{article}

\usepackage[margin=1in]{geometry}
\usepackage{amsmath,amsthm,amssymb,amsfonts}
\usepackage{syntax}
\usepackage{bcprules}
\usepackage{bm}
\usepackage{listings}
\usepackage{wrapfig}
\newtheorem{theorem}{Theorem}
\usepackage{hyperref}

\newcommand{\N}{\mathbb{N}}
\newcommand{\Z}{\mathbb{Z}}
\newcommand{\somany}[1]{$\overline{\mathsf{#1}}$}

\newenvironment{problem}[2][Problem]{\begin{trivlist}
\item[\hskip \labelsep {\bfseries #1}\hskip \labelsep {\bfseries #2.}]}{\end{trivlist}}

\lstset{ %
  basicstyle=\sf,  %
  breaklines=true, %
  columns=fullflexible, %
  literate={<=}{$\le\;$}{1}
  {EMPTY}{$\emptyset\;$}{1}
  {/=>}{$\Downarrow$}{1}
  {|->}{$\mapsto\;$}{1}
  {/\\}{$\;\land\;$}{1},
  morekeywords={if,
    while,
    then,
    else,
    do,
    skip,
    break
  }
}

\begin{document}

\title{CS 565 Spring 2022 Homework 3 (Big-Step Semantics)}
\author{Your name: \underline{\hspace{10cm}}}
\maketitle

Using the big-step operational semantics of \textsc{Imp} (shown in
\autoref{fig:imp+sem}), either (i) fill in the final states produced
by each of the following \textsc{Imp} programs and give a derivation
justifying your answer \textbf{or} (ii) state that the program does
not terminate. \textbf{Note}: You don't need to supply the derivations
for any arithmetic or boolean expressions, just for the \textsc{Imp}
statements.

\begin{problem}{1 (1 point)}
\end{problem}
\vspace{6cm}
\begin{center}
  \lstinline!EMPTY, X := 0; Y := 1 /=> !
\end{center}


\begin{problem}{2 (1 point)}
\end{problem}
\vspace{6cm}
\begin{center}
  \lstinline![X|->2], if (X <= 1) then Y := X + 3 else Z := 4 end /=>!
\end{center}

\pagebreak

\begin{problem}{3 (1 point)}
\end{problem}
\vspace{6cm}
\begin{center}
  \lstinline![X|->0], while (X <= 1) do Y := Y + 1 end /=>!
\end{center}

\begin{problem}{4 (1 point)}
\end{problem}
\begin{center}
  \vspace{14cm}
  \lstinline![Y|->0], while (Y <= 1) do Y := Y + 1 end /=>!
\end{center}


\pagebreak

\begin{problem}{5 (2 points)}
  Several cutting edge languages, including Perl, Visual Basic, and
  Pascal include a \lstinline|repeat c until b| loop construct.  These
  loops differ from the standard \lstinline|while| loops in two ways:
  \begin{enumerate}
  \item the loop guard is checked /after/ the execution of the body, so
      the loop always executes at least once.
    \item the loop continues executing as long as the condition is
      false.
    \end{enumerate}
    Write down the big-step reduction rules for \lstinline|repeat|
    loops in IMP.
  \end{problem}

\pagebreak
\begin{figure}

  \begin{tabular}{lll}
  \begin{minipage}{.25\linewidth}
  \infrule[E-Var]
  {}
  {\sigma, \text{\lstinline|x|} \Downarrow_A \sigma(x)}
\end{minipage}
&
\begin{minipage}{.25\linewidth}
  \infrule[E-Const]
  { }
  {\sigma, \text{\lstinline|n|} \Downarrow_A \text{\lstinline|n|}}
\end{minipage}
    & \begin{minipage}{.45\linewidth}
      \infrule[E-Add]
      {\sigma, \text{\lstinline|a|}_1 \Downarrow_A \text{\lstinline|m|}
        \andalso \text{\lstinline|a|}_2 \Downarrow_A \text{\lstinline|n|}}
      {\sigma, \text{\lstinline|a|}_1 \text{\lstinline|+ a|}_2  \Downarrow_A \text{\lstinline|m + n|}}
\end{minipage} \\\\\hline\\
  \end{tabular}

  \begin{tabular}{lll}
    \begin{minipage}{.3\linewidth}
      \infrule[E-True]
      {}
      {\sigma, \text{\lstinline|true|} \Downarrow_B \text{\lstinline|true|}}
    \end{minipage}
    &
      \begin{minipage}{.3\linewidth}
        \infrule[E-False]
        {}
        {\sigma, \text{\lstinline|false|} \Downarrow_B \text{\lstinline|false|}}
      \end{minipage}
    &
      \begin{minipage}{.4\linewidth}
        \infrule[E-And]
        {\sigma, \text{\lstinline|b|}_1 \Downarrow_B \text{\lstinline|t|}
          \andalso \text{\lstinline|b|}_2 \Downarrow_B \text{\lstinline|v|}}
        {\sigma, \text{\lstinline|b|}_1 \text{\lstinline|/\\ b|}_2  \Downarrow_B \text{\lstinline|t /\\ v|}}
      \end{minipage}
  \end{tabular}

  \begin{tabular}{ll}
    \\
      \begin{minipage}{.4\linewidth}
        \infrule[E-NotT]
        {\sigma, \text{\lstinline|b|} \Downarrow_B \text{\lstinline|true|}}
        {\sigma, \lnot \text{\lstinline|b|} \Downarrow_B \text{\lstinline|false|}}
      \end{minipage}
    &
      \begin{minipage}{.4\linewidth}
        \infrule[E-NotF]
        {\sigma, \text{\lstinline|b|} \Downarrow_B \text{\lstinline|false|}}
        {\sigma, \lnot \text{\lstinline|b|} \Downarrow_B \text{\lstinline|true|}}
      \end{minipage} \\ \\
    \begin{minipage}{.4\linewidth}
      \infrule[E-CmpT]
      {\sigma, \text{\lstinline|a|}_1 \Downarrow_A \text{\lstinline|n|}
        \andalso \sigma, \text{\lstinline|a|}_2 \Downarrow_A \text{\lstinline|n|}}
      {\sigma, \text{\lstinline|a|}_1 \text{\lstinline|= a|}_2  \Downarrow_B \text{\lstinline|true|}}
    \end{minipage} &
        \begin{minipage}{.4\linewidth}
        \infrule[E-CmpF]
        {\sigma, \text{\lstinline|a|}_1 \Downarrow_A \text{\lstinline|m|}
          \andalso \sigma, \text{\lstinline|a|}_2 \Downarrow_A \text{\lstinline|n|}
          \andalso m \neq n}
        {\sigma, \text{\lstinline|a|}_1 \text{\lstinline|= a|}_2  \Downarrow_B \text{\lstinline|true|}}
      \end{minipage} \\\\\hline\\

  \begin{minipage}{.4\linewidth}
  \infrule[E-Skip]
  {}
  {\sigma, \text{\lstinline|skip|} \Downarrow \sigma}
\end{minipage}
&
\begin{minipage}{.4\linewidth}
  \infrule[E-Ass]
  {\sigma, \text{\lstinline|a|} \Downarrow_A \text{\lstinline|n|}}
  {\sigma, \text{\lstinline|x := a|} \Downarrow
    \sigma[\text{\lstinline|x|}\mapsto\text{\lstinline|n|}]}
  \end{minipage}
\\\\
\begin{minipage}{.4\linewidth}
  \infrule[E-Seq]
  {\sigma, \text{\lstinline|c|}_1 \Downarrow \sigma_1
    \andalso \sigma_1, \text{\lstinline|c|}_2 \Downarrow \sigma_2
  }
  {\sigma, \text{\lstinline|c|}_1; \text{\lstinline|c|}_2 \Downarrow
    \sigma_2}
\end{minipage}
&
\begin{minipage}{.45\linewidth}
  \infrule[E-IfTrue]
  {\sigma, \text{\lstinline|b|} \Downarrow_B \text{\lstinline|true|}
    \andalso \sigma, \text{\lstinline|c|}_1 \Downarrow \sigma_1
  }
  {\sigma, \text{\lstinline|if b then c|}_1~ \text{else \lstinline|c|}_2 \Downarrow
    \sigma_1}
\end{minipage}
\\\\
\begin{minipage}{.5\linewidth}
  \infrule[E-IfFalse]
  {\sigma, \text{\lstinline|b|} \Downarrow_B \text{\lstinline|false|}
    \andalso \sigma, \text{\lstinline|c|}_2 \Downarrow \sigma_2
  }
  {\sigma, \text{\lstinline|if b then c|}_1~ \text{else \lstinline|c|}_2 \Downarrow
    \sigma_2}
\end{minipage} &
\begin{minipage}{.45\linewidth}
  \infrule[E-WhileFalse]
  {
    \sigma, \text{\lstinline|b|} \Downarrow_B \text{\lstinline|false|}
  }
  {\sigma, \text{\lstinline|while b do {c}|} \Downarrow \sigma}
\end{minipage}
  \end{tabular}
  \infrule[E-WhileTrue]
  {
    \sigma, \text{\lstinline|b|} \Downarrow_B \text{\lstinline|true|}
      \andalso
      \sigma, \text{\lstinline|c|} \Downarrow \sigma_1
      \andalso
      \sigma_1, \text{\lstinline|while b do {c}|} \Downarrow \sigma_2
    }
    {\sigma, \text{\lstinline|while b do {c}|} \Downarrow \sigma_2}
    \caption{Big-step semantics of arithemetic expressions, boolean
      expressions, and regular \textsc{IMP}.}
    \label{fig:imp+sem}
  \end{figure}
  \pagebreak

\end{document}
