\documentclass[12pt]{article}

\usepackage[margin=1in]{geometry}
\usepackage{amsmath,amsthm,amssymb,amsfonts}
\usepackage{syntax}
\usepackage{bcprules}
\usepackage{mathpartir}
\usepackage{bm}
\usepackage{listings}

\newtheorem{theorem}{Theorem}

\newcommand{\N}{\mathbb{N}}
\newcommand{\Z}{\mathbb{Z}}
\newcommand{\somany}[1]{$\overline{\mathsf{#1}}$}

\newenvironment{problem}[2][Problem]{\begin{trivlist}
\item[\hskip \labelsep {\bfseries #1}\hskip \labelsep {\bfseries #2.}]}{\end{trivlist}}
%If you want to title your bold things something different just make another thing exactly like this but replace "problem" with the name of the thing you want, like theorem or lemma or whatever


\lstset{ %
  basicstyle=\sf,  %
  breaklines=true, %
  columns=fullflexible, %
  literate={<=}{$\le\;$}{1}
  {<->}{$\leftrightarrow\;$}{1}
  {->}{$\rightarrow$}{1}
  {|-}{$\vdash$}{1}
  {<>}{$\neq\;$}{1}
  {\\.}{$\lambda\;$}{1}
  {/\\}{$\;\land\;$}{1},
  morekeywords={if,
    while,
    then,
    else,
    do,
    skip,
    break
  }
}

\begin{document}

\title{CS 565 Spring 2022 Homework 6 \\ (Type Inference + Subtyping)}
\author{Your name: \underline{Shuang Wu (wu1716@purdue.edu)}}
\maketitle

\begin{problem}{1 (1 point)}
  Construct a constraint typing derivation whose conclusion is
  \[
    \vdash \mathsf{\lambda x : X.\; \lambda y : Y.\; \lambda z : Z.\; (x\;
      z)\; (y\; z) : S ~\mid~\mathcal{C}}
  \]
  for some $\mathsf{S}$, $\mathcal{C}$.

\begin{center}
  \begin{mathpar}
  \inferrule*[Right=CTAbs]
  {
    \inferrule*[Right=CTAbs]
    {
      \inferrule*[Right=CTAbs]
      {
        \inferrule*[Right=CTApp]
        {
          \inferrule*[Right=CTApp]
          {
            \inferrule*[Left=CTVar]
            {[x \mapsto X, y \mapsto Y, z \mapsto Z](x) : X}
            {[x \mapsto X, y \mapsto Y, z \mapsto Z] \vdash x : X ~\mid~ \emptyset}
            \\
            \inferrule*[Right=CTVar]
            {[x \mapsto X, y \mapsto Y, z \mapsto Z](z) : Z}
            {[x \mapsto X, y \mapsto Y, z \mapsto Z] \vdash z : Z ~\mid~ \emptyset}
          }
          {[x \mapsto X, y \mapsto Y, z \mapsto Z] \vdash (x \; z) : T_{xz} ~\mid~ \{X = Z \rightarrow T_{xz}\}}
          \\
          \inferrule*[Right=CTApp]
          {
            \inferrule*[Left=CTVar]
            {[x \mapsto X, y \mapsto Y, z \mapsto Z](y) : Y}
            {[x \mapsto X, y \mapsto Y, z \mapsto Z] \vdash y : Y ~\mid~ \emptyset}
            \\
            \inferrule*[Right=CTVar]
            {[x \mapsto X, y \mapsto Y, z \mapsto Z](z) : Z}
            {[x \mapsto X, y \mapsto Y, z \mapsto Z] \vdash z : Z ~\mid~ \emptyset}
          }
          {[x \mapsto X, y \mapsto Y, z \mapsto Z] \vdash (y \; z) : T_{yz} ~\mid~ \{Y = Z \rightarrow T_{yz}\}}
        }
        {[x \mapsto X, y \mapsto Y, z \mapsto Z] \vdash (x \; z) \; (y \; z) : T ~\mid~ \mathcal{C}}
      }
      {[x \mapsto X, y \mapsto Y] \vdash \lambda z : Z . \; (x \; z) \; (y \; z) : Z \rightarrow T ~\mid~ \mathcal{C}}
    }
    {[x \mapsto X] \vdash \lambda y : Y . \; \lambda z : Z . \; (x \; z) \; (y \; z) : Y \rightarrow Z \rightarrow T ~\mid~ \mathcal{C}}
  }
  {\vdash \lambda x : X . \; \lambda y : Y . \; \lambda z : Z . \; (x \; z) \; (y \; z) : S ~\mid~ \mathcal{C}}
  \end{mathpar}
\end{center}

$\mathcal{C} \equiv \{X = Z \rightarrow T_{xz}, Y = Z \rightarrow T_{yz}, T_{xz} = T_{yz} \rightarrow T\}$

$ S \equiv X \rightarrow Y \rightarrow Z \rightarrow T$

\end{problem}

\begin{problem}{2 (2 points)}
  Write down principal unifers (when they exist) for the following
  sets of constraints:
  \begin{itemize}

  \item $\mathsf{\{\}}$ (The empty set of constraints) $\equiv \{\}$

  \item $\mathsf{\{Y = V \rightarrow U, Y = X \rightarrow V\}} \equiv \mathsf{\{X \rightarrow X = X \rightarrow X, X \rightarrow X = X \rightarrow X\}}$

  \item $\mathsf{\{X = Bool, Y = X \rightarrow X\}} \equiv \mathsf{\{Bool = Bool, Bool \rightarrow Bool = Bool \rightarrow Bool\}}$

  \item $\mathsf{\{ Bool \rightarrow Bool = X \rightarrow Y \}} \equiv \mathsf{\{Bool \rightarrow Bool = Bool \rightarrow Bool\}}$

  \item $\mathsf{\{ (Bool \rightarrow Y) \rightarrow Bool = Bool \rightarrow U \}}$ fail to unify.

  \end{itemize}
\end{problem}

\begin{problem}{3 (2 points)}
  Suppose we have types \lstinline|S, T, U|, and \lstinline|V| with
  \lstinline|S <: T| and \lstinline|U <: V|.  Which of the following
  subtyping assertions are then true? Write true or false after each
  one.

  \begin{itemize}
  \item \lstinline|T->S <: T->S| True.

  \item \lstinline|T->T->U <: S->S->V| True.

  \item \lstinline|(T->T)->U <: (S->S)->V| False.

  \item \lstinline|((T->S)->T)->U <: ((S->T)->S)->V| True.

  \end{itemize}

\end{problem}

\pagebreak
\begin{problem}{4 (1 point)}
  How many supertypes does the type
  \begin{itemize}
  \item [ ] \lstinline|{{x: {z:Bool, q: Nat}, y: Bool -> Bool}}|
  \end{itemize}
  have? That is, how many different types \lstinline|T| are there such
  that
  \begin{itemize}
  \item[ ] \lstinline|{x: {z:Bool, q: Nat}, y: Bool -> Bool} <: T|
  \end{itemize}
  (We consider two types to be
  different if they are written differently, even if each is a subtype
  of the other. For example, \lstinline|{x:A,y:B}| and
  \lstinline|{y:B,x:A}| are different.)
  
  The record has 17 supertypes:
  
  $\{\}$,
  
  $\{y: \text{Bool} \rightarrow \text{Bool}\}$,
  
  $\{x: \{\}\}$,
  
  $\{x: \{z: \text{Bool}\}\}$,
  
  $\{x: \{q: \text{Nat}\}\}$,
  
  $\{x: \{z: \text{Bool}, q: \text{Nat}\}\}$,
  
  $\{x: \{q: \text{Nat}, z: \text{Bool}\}\}$,
  
  $\{x: \{\}, y: \text{Bool} \rightarrow \text{Bool}\}$,
  
  $\{x: \{z: \text{Bool}\}, y: \text{Bool} \rightarrow \text{Bool}\}$,
  
  $\{x: \{q: \text{Nat}\}, y: \text{Bool} \rightarrow \text{Bool}\}$,
  
  $\{x: \{z: \text{Bool}, q: \text{Nat}\}, y: \text{Bool} \rightarrow \text{Bool}\}$,
  
  $\{x: \{q: \text{Nat}, z: \text{Bool}\}, y: \text{Bool} \rightarrow \text{Bool}\}$
  
  $\{y: \text{Bool} \rightarrow \text{Bool}, x: \{\}\}$,
  
  $\{y: \text{Bool} \rightarrow \text{Bool}, x: \{z: \text{Bool}\}\}$,
  
  $\{y: \text{Bool} \rightarrow \text{Bool}, x: \{q: \text{Nat}\}\}$,
  
  $\{y: \text{Bool} \rightarrow \text{Bool}, x: \{z: \text{Bool}, q: \text{Nat}\}\}$,
  
  $\{y: \text{Bool} \rightarrow \text{Bool}, x: \{q: \text{Nat}, z: \text{Bool}\}\}$
  
\end{problem}

\begin{problem}{5 (2 points)}
  The subtyping rule for product types:

  \infrule{\mathsf{S_1 <: T_1}  \andalso  \mathsf{S_2 <: T_2}}
  { \mathsf{S_1*S_2 <: T_1*T_2}}

\noindent intuitively corresponds to the ``depth'' subtyping rule for
  records. Extending the analogy, a language designer might consider
  including a ``permutation'' rule as well

  \infrule{}{\mathsf{T_1*T_2 <: T_2*T_1}}

\noindent for products. Explain in a couple of sentences why such a subtyping
  rule is or is not sound?
  
  This "permutation" rule is not sound. For instance, assume $T_1 \equiv \text{Nat}$ and $T_2 \equiv \text{Bool}$, $\text{Nat} * \text{Bool} <: \text{Bool} * \text{Nat}$ means any value of type $\text{Nat} * \text{Bool}$ must be usable in every way a $\text{Bool} * \text{Nat}$ is. However, this does not hold for all values. Consider the case that $v_1 = (1, \text{true})$ and $v_2 = (\text{true}, 1)$. $v_2$ is usable in
  
  \begin{center}
    $\textbf{if} ~\text{fst}~ v_2 ~\textbf{then}~ \text{true} ~\textbf{else}~ \text{false}$
  \end{center}
  
\noindent but $v_1$ is not.
\end{problem}

\end{document}
