\documentclass[12pt]{article}

\usepackage[margin=1in]{geometry}
\usepackage{amsmath,amsthm,amssymb,amsfonts}
\usepackage{syntax}
\usepackage{bcprules}
\usepackage{mathpartir}
\usepackage{bm}
\usepackage{listings}
\usepackage{wrapfig}
\newtheorem{theorem}{Theorem}
\usepackage{hyperref}

\newcommand{\N}{\mathbb{N}}
\newcommand{\Z}{\mathbb{Z}}
\newcommand{\somany}[1]{$\overline{\mathsf{#1}}$}

\newenvironment{problem}[2][Problem]{\begin{trivlist}
\item[\hskip \labelsep {\bfseries #1}\hskip \labelsep {\bfseries #2.}]}{\end{trivlist}}

\lstset{ %
  basicstyle=\sf,  %
  breaklines=true, %
  columns=fullflexible, %
  literate={<=}{$\le\;$}{1}
  {<->}{$\leftrightarrow\;$}{1}
  {->}{$\rightarrow$}{1}
  {|-}{$\vdash$}{1}
  {<>}{$\neq\;$}{1}
  {\\.}{$\lambda\;$}{1}
  {/\\}{$\;\land\;$}{1},
  morekeywords={if,
    while,
    then,
    else,
    do,
    skip,
    break
  }
}

\begin{document}

\title{CS 565 Spring 2022 Homework 5 \\ (Axiomatic Semantics Again)}
\author{Your name: \underline{Shuang Wu (wu1716@purdue.edu)}}
\maketitle

\begin{problem}{1 (2 points)}
What weakest precondition does the \lstinline|wp| function compute for
each of the following programs and postconditions:

\begin{itemize}
\item[a. ]   \lstinline!wp( X := X + 1, {X > 5})!

$\equiv$ \lstinline!{X + 1 > 5} = {X > 4}!

\item[b. ]    \lstinline!wp( X := Y, {X <> Y})!

$\equiv$ \lstinline!{Y <> Y} = {False}!

\item[c. ]   \lstinline!wp( X := X + 1; Y := Z, {X = Y})!

$\equiv$ \lstinline!{X + 1 = Z}!

\item[d. ]   \lstinline!wp( if (X < 10) then Z := X else Z := Y, {Z < 10})!

$\equiv$ \lstinline!X < 10 -> {X < 10} /\ X! $\nless$ 10 \lstinline!-> {Y < 10}!

\lstinline!= X < 10 -> {X < 10} /\ 10 <= X -> {Y < 10}!

\end{itemize}
\end{problem}

\pagebreak

\begin{problem}{2 (1 point)}
  Using the typing rules from \autoref{fig:typing+STLC}, provide a
  derivation tree showing that the following program has the indicated
  type:
  \begin{center}
    \begin{mathpar}
      \inferrule*[Right=T-App]
      {
        \inferrule*[Right=T-Var]
        {[f: \text{Bool} \rightarrow \text{Bool}] (f) = \text{Bool} \rightarrow \text{Bool}}
        {f: \text{Bool} \rightarrow \text{Bool} \vdash f: \text{Bool} \rightarrow \text{Bool}}
        \\
        \inferrule*[Right=T-If]
        {
          \inferrule*[Right=T-False] { }
          {f: \text{Bool} \rightarrow \text{Bool} \vdash \text{false}: \text{Bool}}
          \\
          \inferrule*[Left=T-True] { }
          {f: \text{Bool} \rightarrow \text{Bool} \vdash \text{true}: \text{Bool}}
          \\
          \inferrule*[Right=T-False] { }
          {f: \text{Bool} \rightarrow \text{Bool} \vdash \text{false}: \text{Bool}}
        }
        {f: \text{Bool} \rightarrow \text{Bool} \vdash \textbf{if}~ \text{false}~ \textbf{then}~ \text{true}~ \textbf{else}~ \text{false}: \text{Bool}}
      }
      {f: \text{Bool} \rightarrow \text{Bool} \vdash f (\textbf{if}~ \text{false}~ \textbf{then}~ \text{true}~ \textbf{else}~ \text{false}): \text{Bool}}
    \end{mathpar}
  \end{center}
\end{problem}

\begin{problem}{3 (1 point)}
  Using the typing rules from \autoref{fig:typing+STLC}, provide a
  derivation tree showing that the following program has the indicated
  type:
  \begin{center}
    \begin{mathpar}
      \inferrule*[Right=T-Abs]
      {
        \inferrule*[Right=T-App]
        {
          \inferrule*[Right=T-Var]
          {[f: \text{Bool} \rightarrow \text{Bool}~ ; x: \text{Bool}] (f) = \text{Bool} \rightarrow \text{Bool}}
          {f: \text{Bool} \rightarrow \text{Bool}~ ; x: \text{Bool} \vdash f: \text{Bool} \rightarrow \text{Bool}}
          \\
          \inferrule*[Right=T-If]
          {
            \inferrule*[Right=T-Var]
            {[f: \text{Bool} \rightarrow \text{Bool}~ ; x: \text{Bool}](x) = \text{Bool}}
            {f: \text{Bool} \rightarrow \text{Bool}~ ; x: \text{Bool} \vdash x: \text{Bool}}
            \\
            \inferrule*[Left=T-True] { }
            {f: \text{Bool} \rightarrow \text{Bool}~ ; x: \text{Bool} \vdash \text{true}: \text{Bool}}
            \\
            \inferrule*[Right=T-False] { }
            {f: \text{Bool} \rightarrow \text{Bool}~ ; x: \text{Bool} \vdash \text{false}: \text{Bool}}
          }
        {f: \text{Bool} \rightarrow \text{Bool}~ ; x: \text{Bool} \vdash \textbf{if}~ x~ \textbf{then}~ \text{true}~ \textbf{else}~ \text{false}: \text{Bool}}
        }
        {f: \text{Bool} \rightarrow \text{Bool}~ ; x: \text{Bool} \vdash f (\textbf{if}~ x~ \textbf{then}~ \text{false}~ \textbf{else}~ \text{true}) : \text{Bool}}
      }
      {f: \text{Bool} \rightarrow \text{Bool} \vdash \lambda~ x: \text{Bool} .~ f (\textbf{if}~ x~ \textbf{then}~ \text{false}~ \textbf{else}~ \text{true}): \text{Bool} \rightarrow \text{Bool}}
    \end{mathpar}
  \end{center}
\end{problem}

\begin{problem}{4 (1 point)}
  Using the typing rules from \autoref{fig:typing+STLC}, provide a
  context $\Gamma$ under which the term \lstinline!f x y!  has type
  bool. Can you give a simple and informal description of the set of
  \emph{all} such contexts?
  
  \begin{center}
      $f : \text{nat} \rightarrow \text{nat} \rightarrow \text{bool}~;~ x : \text{nat}~;~ y : \text{nat}$
  \end{center}
  
  As long as the last type (return type) of $f$ is bool, other types are unrestricted. Any context fit this predicate falls into the set.
  
\end{problem}

\pagebreak

\begin{figure}

  \infrule[T-Var]
  { \Gamma(x) = T   }
    {\Gamma \vdash x : T }
\vspace{1cm}

\infrule[T-Abs]
  {\Gamma, x : T_1 \vdash t ~:~ T_2  }
  {\Gamma \vdash \lambda\;x:T_1.~ t ~:~ T_1 \rightarrow T_2 }
  \vspace{1cm}

  \infrule[T-Abs]
  { \Gamma \vdash t_1 ~:~ T_2 \rightarrow T_1
    \andalso \Gamma \vdash t_2 ~:~ T_2
  }
  {\Gamma \vdash t_1~t_2 ~:~ T_1 }
  \vspace{1cm}

  \infrule[T-True]
  {}
  { \Gamma \vdash \text{\lstinline|true|} ~:~ \text{\lstinline|Bool|}}
  \vspace{1cm}

  \infrule[T-False]
  {}
  { \Gamma \vdash \text{\lstinline|false|} ~:~ \text{\lstinline|Bool|}}
  \vspace{1cm}

  \infrule[T-If]
  {\Gamma \vdash t_1 ~:~ \text{\lstinline|Bool|}
      \andalso \Gamma \vdash t_2 ~:~ T
      \andalso \Gamma \vdash t_3 ~:~ T}
  { \Gamma \vdash \text{\lstinline|if|} ~t_1~ \text{\lstinline|then|}
    ~t_2~ \text{\lstinline|else|} ~t_3~:~ T}


    \caption{Typing rules for the simply typed lambda calculus with
      booleans from the STLC chapter of PLF.}
    \label{fig:typing+STLC}
  \end{figure}



\end{document}
